%% Template for Master thesis
%% ===========================
%%
%% You need at least KomaScript v3.0.0,
%% e.g. available in Texlive 2009
\documentclass[12pt, a4paper, twoside]{book}
\usepackage[a4paper,left=3cm,right=2cm,top=2.5cm,bottom=2.5cm,bindingoffset=5mm]{geometry}
\usepackage{setspace}

% used pagages
\usepackage[utf8]{inputenc}
\usepackage[T1]{fontenc}
\usepackage[english]{babel}

\usepackage{amsmath}
\usepackage{amssymb}
\usepackage{amsfonts}
\usepackage{slashed}

\usepackage{graphicx}

\usepackage{cite}
\bibliographystyle{abbrv}
\usepackage{hyperref}

\usepackage{appendix}

% german date setup
\usepackage[ngerman]{datetime}
\newdateformat{myformat}{\THEDAY{ten }\monthnamengerman[\THEMONTH], \THEYEAR}

% colours
\usepackage{color}
\definecolor{darkblue}{rgb}{0.0,0.0,0.4}
\definecolor{darkgreen}{rgb}{0.0,0.4,0.0}

% links
\hypersetup{
    colorlinks,
    linkcolor=black,
    citecolor=darkgreen,
    urlcolor=darkblue
}

\newcommand{\brac}[1] {\!\left(#1\right)}

\begin{document}
\pagestyle{empty}


%% this will generate title pages similar to the template provided
%% by the Department of Physics and Astronomy Heidelberg
%%
%% More information:
%% http://www.physik.uni-heidelberg.de/aktuelles/studium/
%% (PDF link: ...studium/download/145/Vorlage_Diplomarbeit_Formular.pdf)

%% Titleintro
\thispagestyle{empty}
\begin{center}
  \renewcommand{\baselinestretch}{2.00}
  \Large\sffamily
  Department of Physics and Astronomy\\
  \large University of Heidelberg
  \par\vfill\normalfont
  Master thesis\\
  in Physics\\
  submitted by\\
  David Conway Lafferty\\
  born in Glasgow\\
  1994
\end{center}
\cleardoublepage

%% Titlepage
\thispagestyle{empty}
\begin{center}
  \renewcommand{\baselinestretch}{2.00}
  \Large\bfseries\sffamily
    (Title)\\
    (of)\\
    (Master thesis)
  \par
  \vfill
  \large\normalfont
  This Master thesis has been carried out by David Conway Lafferty\\
  at the\\
  Institute of Theoretical Physics\\
  under the supervision of\\
  Prof. Dr. Jan M. Pawlowski\\
  \& \\
  Dr. Alexander K. Rothkopf\\
  %% additionally insert second supervisor here if carrying out an
  %% external diploma thesis. Reduce vspace in L. 44 accordingly.
\end{center}\par
\vspace{5\baselineskip}

% reset baselinestretch
\renewcommand{\baselinestretch}{1.00}\normalsize
\cleardoublepage

\cleardoublepage

\pagenumbering{roman}
\pagestyle{plain}

\include{abstracts}

\cleardoublepage

\cleardoublepage

\tableofcontents

\cleardoublepage

\pagenumbering{arabic}
\chapter{Introduction}
\onehalfspacing 
The strong interaction is the strongest of the four fundamental forces of nature. It is described by quantum chromodynamics (QCD), a quantum field theory exhibiting many peculiar properties. The first, known as asymptotic freedom, is that the underlying interaction strength of QCD decreases as the relevant energy scale increases. Another, which is still not completely understood, is colour confinement -- the phenomenon that the fundamental degrees of freedom of QCD, quarks and gluons, do not exist as isolated objects and instead form bound states known as hadrons. Hadrons make up most of the matter we experience in our everyday lives, and thus colour confinement is observed ubiquitously at the rather mundane energy scales that are naturally present on Earth. However, a more exotic state of matter is theorised to exist at extremely high temperatures or densities -- the Quark Gluon Plasma (QGP). In the QGP, quarks and gluons are considered as being asymptotically free and no longer confined to within the bounds of a hadron. More generally speaking, the QGP is expected to be one of many regions in the entire phase space of strongly interacting matter. A schematic phase diagram is shown in Fig.\ref{fig:phase}, where one can see for example the location of neutron stars at high density and low temperature. Indeed, the QGP itself is believed to have existed in the early moments of our universe, and thus understanding its properties will form a crucial part of answering some of the deepest questions of human thought.  

The monumental experimental effort aimed at detecting and quantifying the QGP has culminated today in the relativistic heavy-ion colliders such at those at BNL, CERN, and GSI. The complexity of such experiments has necessitated the development of new techniques both in experiment and theory, in order to firstly map the measured experimental data to QGP properties (a highly non-trivial process) and then to understand how these macroscopic properties emerge from the underlying microscopic theory of QCD. With regards to the former, one refers to various ``probes'' that may indicate the presence of QGP formation. This thesis revolves around one such probe, namely heavy quarkonium.

The bound states of a heavy quark and anti-quark of the same flavour are known generically as quarkonia. Since the seminal work of Matsui and Statz [REF], the interest in quarkonium as a probe of the QGP has grown into a considerable subfield in the realm of heavy ion collisions. From an experimental perspective, an intricate and not yet fully understood structure has emerged in the production and decay of these mesons throughout the collision process. From the theory side, the development of new effective field theories [REF] has allowed quantitative predictions to be made from ever more rigorous formalisms. One such formalism, known as pNRQCD, relies on separating the typical scales present in the system, so that the dynamics of the bound state are governed by an effective potential in a non-relativistic Schro{\"o}dinger equation [REF]. In this way, the complexities of the full quantum field theory are reduced to a much more tractable quantum mechanical problem.

This thesis presents a new prescription for parametrising the static heavy-quark potential in a background of hot and deconfined charge carriers, such as the QGP. By generalising the Gauss law of classical electromagnetism and combining this with a field-theoretic in-medium permittivity, the resulting in-medium complex potential admits an analytical solution. This can then be used to calculate spectral functions, and give realistic phenomenological predictions. The outline of this thesis is as follows: in Chapter 2, we give a short summary of some theoretical aspects of QCD, as well as an introduction into quarkonium phenomenology both in vacuum and in the context of heavy ion collisions. Chapter 3 provides a detailed derivation of the in-medium potential and shows that this parametrisation is able to faithfully reproduce lattice data by utilising only one fitting parameter, the inverse screening length. Chapter 4 outlines the procedure with which phenomenologically relevant quantities such as the melting temperatures, decay widths, and electromagnetic decay ratios can be calculated. The main results of this thesis are also given here, and a comparison is made with recent experimental results. A summary and outlook is given in Chapter 5. Appendix A contains a short introduction to thermal field theory and in particular the notion of a spectral function. Appendix B gives a more formal derivation of the Debye mass at one-loop order via Euclidean thermal field theory and finally, Appendix C shows how the structure of the in-medium permittivity arises from the Schwinger-Keldysh formalism. 

\chapter{Theory overview}
\onehalfspacing
In this Chapter, we provide the theoretical foundations of various topics that will be important throughout the rest of this thesis. We start with an introduction into quantum chromodynamics, by constructing the Lagrangian and looking at the fundamental interactions. Some more details will be given on the phenomena of confinement and asymptotic freedom, before a brief discussion of QCD thermodynamics and the phase diagram. The presentation will mainly follow [REF] and [REF]. We then give an overview of vacuum heavy quarkonium physics by detailing some of the techniques used to discern for example the numerous quantum states in the heavy quark anti-quark system. Finally, we introduce some important concepts in heavy ion collisions, with an emphasis on heavy quarkonium phenomenology. 
\section{Aspects of QCD}
In the 1970s, Murray Gell-Mann and George Zweig independently proposed a model to explain the observed spectrum of strongly interacting particles that contained the idea of \emph{quarks} as elementary particles of fractional charge that exist within hadrons. While explaining and predicting some aspects very well, the quark model contained notable flaws. Namely, the lack of observation of free particles with fractional charge, and the existence of some states apparently in violation of the well-established exclusion principle that quarks, as fermions, must obey. The resolution of the first problem was the introduction of a new, additional quantum number termed \emph{colour}. Each quark would carry one of three possible colour charges -- red, green, or blue, -- the symmetry properties of which mitigated the existence of the problematic states. The second problem was solved by the discovery that non-Abelian gauge theories exhibit asymptotic freedom [REF], which allowed the theory of strong interactions to be brought into its final form. Namely, quantum chromodynamics as a non-Abelian gauge theory with colour symmetry group \(SU\brac{3}\), coupled to quarks acting under the fundamental representation. As is often the case in science, we have the luxury of summarising decades of previous generations' work in a mere few sentences, glossing over the murky and often enlightening details. For a detailed historical account of the development of QCD, the reader can consult [REF].  

As a quantum field theory, the fundamental object of QCD is its Lagrangian density (often denoted simply as the Lagrangian), \(\mathcal{L}_{QCD}\), which we now proceed to construct based on the guiding properties outlined in the previous paragraph. The quarks and anti-quarks are described respectively by the Dirac spinor fields 
\begin{equation}
\psi_{\alpha,i,f}\brac{x}, \quad\quad \bar{\psi}_{\alpha,i,f}\brac{x},
\end{equation} 
where \(\alpha\) is the spinor index representing the underlying Poincar\'e invariance, \(i=1,2,3\) is the colour index, \(f=1\ldots N_{F}\) labels the flavour quantum number (\(f=\) up, down, strange, charm, bottom, top), and \(x\) is the position 4-vector. The free quark Lagrangian is then
\begin{equation}
\label{eq:quarklag}
\mathcal{L}_{quark}=\sum_{f}\sum_{i}\bar{\psi}_{\alpha,i,f}\brac{x}\brac{i\brac{\gamma^{\mu}\partial_{\mu}}_{\alpha\beta}-\delta_{\alpha\beta}m_f}\psi_{\beta,i,f}\brac{x},
\end{equation}
where \(m_{f}\) is the quark mass. At this point, we impose local gauge invariance as necessitated by the non-Abelian nature of the theory. That is, we require the Lagrangian to remain invariant under the following transformation (neglecting indices):
\begin{equation}
\psi '\brac{x}=U\brac{x}\psi\brac{x},\quad\quad\bar{\psi}'\brac{x}=\bar{\psi}\brac{x}U^\dagger\brac{x},
\end{equation}
where the transformation matrix 
\begin{equation}
U\brac{x}=e^{i\epsilon\left(x\right)}=e^{i\sum\epsilon_a\brac{x} t_a}
\end{equation}
with group parameters \(\varepsilon_a\brac{x}\) and generators \(t_a\) acts on the colour indices. One can easily verify that the mass term in Eq.~\eqref{eq:quarklag} remains invariant, however the spacetime derivative in the kinetic term does not. The resolution is to promote the 4-derivative to the covariant derivative 
\begin{equation}
D_\mu=\partial_\mu-igA_\mu
\end{equation}
where the gauge field \(A^\mu\brac{x}=\sum_a A^\mu_a\brac{x}t_a\) is associated to the force-mediating bosons (gluons) and \(g\) is the coupling strength.  
\section{Quarkonium in vacuum}
\section{}


\chapter{The in-medium potential}
\onehalfspacing
\chapter{Application to Heavy Ion Collisions}
\onehalfspacing
\chapter{Conclusion}
\onehalfspacing

\begin{appendices}
\chapter{Debye mass derivation}
\chapter{Schwinger-Keldysh formalism}
\end{appendices}

\bibliography{references}

\chapter*{Acknowledgements}


\cleardoublepage
\include{deposition}

\end{document}
