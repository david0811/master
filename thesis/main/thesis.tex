%% Template for Master thesis
%% ===========================
%%
%% You need at least KomaScript v3.0.0,
%% e.g. available in Texlive 2009
\documentclass[12pt, a4paper, twoside]{book}
\usepackage[a4paper,left=3cm,right=2cm,top=2.5cm,bottom=2.5cm,bindingoffset=5mm]{geometry}
\usepackage{setspace}

% used pagages
\usepackage[utf8]{inputenc}
\usepackage[T1]{fontenc}
\usepackage[english]{babel}

\usepackage{amsmath}
\usepackage{amssymb}
\usepackage{amsfonts}

\usepackage{graphicx}

\usepackage{cite}
\bibliographystyle{abbrv}
\usepackage{hyperref}

\usepackage{appendix}

% german date setup
\usepackage[ngerman]{datetime}
\newdateformat{myformat}{\THEDAY{ten }\monthnamengerman[\THEMONTH], \THEYEAR}

% colours
\usepackage{color}
\definecolor{darkblue}{rgb}{0.0,0.0,0.4}
\definecolor{darkgreen}{rgb}{0.0,0.4,0.0}

% links
\hypersetup{
    colorlinks,
    linkcolor=black,
    citecolor=darkgreen,
    urlcolor=darkblue
}

\begin{document}
\pagestyle{empty}


%% this will generate title pages similar to the template provided
%% by the Department of Physics and Astronomy Heidelberg
%%
%% More information:
%% http://www.physik.uni-heidelberg.de/aktuelles/studium/
%% (PDF link: ...studium/download/145/Vorlage_Diplomarbeit_Formular.pdf)

%% Titleintro
\thispagestyle{empty}
\begin{center}
  \renewcommand{\baselinestretch}{2.00}
  \Large\sffamily
  Department of Physics and Astronomy\\
  \large University of Heidelberg
  \par\vfill\normalfont
  Master thesis\\
  in Physics\\
  submitted by\\
  David Conway Lafferty\\
  born in Glasgow\\
  1994
\end{center}
\cleardoublepage

%% Titlepage
\thispagestyle{empty}
\begin{center}
  \renewcommand{\baselinestretch}{2.00}
  \Large\bfseries\sffamily
    (Title)\\
    (of)\\
    (Master thesis)
  \par
  \vfill
  \large\normalfont
  This Master thesis has been carried out by David Conway Lafferty\\
  at the\\
  Institute of Theoretical Physics\\
  under the supervision of\\
  Prof. Dr. Jan M. Pawlowski\\
  \& \\
  Dr. Alexander K. Rothkopf\\
  %% additionally insert second supervisor here if carrying out an
  %% external diploma thesis. Reduce vspace in L. 44 accordingly.
\end{center}\par
\vspace{5\baselineskip}

% reset baselinestretch
\renewcommand{\baselinestretch}{1.00}\normalsize
\cleardoublepage

\cleardoublepage

\pagenumbering{roman}
\pagestyle{plain}

%% Abstract page
%% =============
%%
%% Content of abstract pages has been put into seperate pages to simplify
%% word counting. Use e.g. the unix command
%%   wc abstract-ger.tex
%% or
%%   wc abstract-eng.tex
%% to get the number of words contained in these files.
\thispagestyle{empty}
\begin{center}
  \begin{minipage}[c][0.48\textheight][b]{0.9\textwidth}
    \small
    \textbf{
      (Titel der Masterarbeit - deutsch):
    }\par
    \vspace{\baselineskip}
    %% Latex markup und Zitate funktionieren auch hier
(Abstract in Deutsch, max. 200 Worte. Beispiel: \cite{loremIpsum})

Lorem ipsum dolor sit amet, consectetur adipisici elit, sed eiusmod tempor
incidunt ut labore et dolore magna aliqua. Ut enim ad minim veniam, quis
nostrud exercitation ullamco laboris nisi ut aliquid ex ea commodi consequat.
Quis aute iure reprehenderit in voluptate velit esse cillum dolore eu fugiat
nulla pariatur. Excepteur sint obcaecat cupiditat non proident, sunt in culpa
qui officia deserunt mollit anim id est laborum.

Duis autem vel eum iriure dolor in hendrerit in vulputate velit esse molestie
consequat, vel illum dolore eu feugiat nulla facilisis at vero eros et
accumsan et iusto odio dignissim qui blandit praesent luptatum zzril delenit
augue duis dolore te feugait nulla facilisi. Lorem ipsum dolor sit amet,
consectetuer adipiscing elit, sed diam nonummy nibh euismod tincidunt ut
laoreet dolore magna aliquam erat volutpat.

Ut wisi enim ad minim veniam, quis nostrud exerci tation ullamcorper suscipit
lobortis nisl ut aliquip ex ea commodo consequat. Duis autem vel eum iriure
dolor in hendrerit in vulputate velit esse molestie consequat, vel illum dolore
eu feugiat nulla facilisis at vero eros et accumsan et iusto odio dignissim qui
blandit praesent luptatum zzril delenit augue duis dolore te feugait nulla
facilisi.
  \end{minipage}\par
  \vfill
  \begin{minipage}[c][0.48\textheight][b]{0.9\textwidth}
    \small
    \textbf{
      (Title of Master thesis - english):
    }\par
    \vspace{\baselineskip}
    %% Latex markup and citations may be used here
Lorem ipsum dolor sit amet, consectetur adipisici elit, sed eiusmod tempor
incidunt ut labore et dolore magna aliqua. Ut enim ad minim veniam, quis
nostrud exercitation ullamco laboris nisi ut aliquid ex ea commodi consequat.
Quis aute iure reprehenderit in voluptate velit esse cillum dolore eu fugiat
nulla pariatur. Excepteur sint obcaecat cupiditat non proident, sunt in culpa
qui officia deserunt mollit anim id est laborum.

Duis autem vel eum iriure dolor in hendrerit in vulputate velit esse molestie
consequat, vel illum dolore eu feugiat nulla facilisis at vero eros et
accumsan et iusto odio dignissim qui blandit praesent luptatum zzril delenit
augue duis dolore te feugait nulla facilisi. Lorem ipsum dolor sit amet,
consectetuer adipiscing elit, sed diam nonummy nibh euismod tincidunt ut
laoreet dolore magna aliquam erat volutpat.

Ut wisi enim ad minim veniam, quis nostrud exerci tation ullamcorper suscipit
lobortis nisl ut aliquip ex ea commodo consequat. Duis autem vel eum iriure
dolor in hendrerit in vulputate velit esse molestie consequat, vel illum dolore
eu feugiat nulla facilisis at vero eros et accumsan et iusto odio dignissim qui
blandit praesent luptatum zzril delenit augue duis dolore te feugait nulla
facilisi.
  \end{minipage}
\end{center}


\cleardoublepage

\cleardoublepage

\tableofcontents

\cleardoublepage

\pagenumbering{arabic}
\chapter{Introduction}
\onehalfspacing 
The strong interaction is the strongest of the four fundamental forces of nature. It is described by Quantum Chromodynamics (QCD), a quantum field theory exhibiting many peculiar properties. The first, known as asymptotic freedom, is that the underlying interaction strength in QCD decreases as the energy scale of the system increases. Another, which is still not completely understood, is colour confinement -- the phenomenon that the fundamental degrees of freedom of QCD, quarks and gluons, do not exist as isolated objects and instead form bound states known as hadrons. Hadrons make up most of the matter we experience in our everyday lives, and thus colour confinement is observed ubiquitously at the rather mundane energy scales that are naturally present on Earth. However, a more exotic state of matter is theorised to exist at extremely high temperatures or densities -- the Quark Gluon Plasma (QGP). In the QGP, quarks and gluons are considered as being asymptotically free and no longer confined to within the bounds of a hadron. More generally speaking, the QGP is expected to be one of many regions in the entire phase space of strong interacting matter. A schematic phase diagram is shown in Fig.\ref{fig:phase}, where one can see for example the location of neutron stars at high density and low temperature. Indeed, the QGP itself is believed to have existed in the early moments of our universe, and thus understanding its properties will form a crucial part of answering some of the deepest questions of human thought.  

The monumental experimental effort aimed at detecting and quantifying the QGP has culminated today in the relativistic heavy-ion colliders such at those at BNL, CERN, and GSI. The complexity of such experiments has necessitated the development of new techniques both in experiment and theory, in order to firstly map the measured experimental data to QGP properties (a highly non-trivial process) and then to understand how these macroscopic properties emerge from the underlying microscopic theory of QCD. With regards to the former, one refers to various ``probes'' that may indicate the presence of QGP formation. This thesis revolves around one such probe, namely heavy quarkonium.

The bound states of a heavy quark and anti-quark of the same flavour are known generically as quarkonia. Since the original suggestion by Matsui and Statz [REF], the interest in quarkonium as a probe of the QGP has grown into a considerable subfield in the realm of heavy ion collisions. From an experimental perspective, an intricate and not yet fully understood structure has emerged in the production and decay of these mesons throughout the collision process. From the theory side, the development of new effective field theories [REF] has allowed quantitative predictions to be made from ever more rigorous formalisms. One such formalism, known as pNRQCD, relies on separating the typical scales present in the system, and allows the dynamics of the bound state to be governed by an effective potential in non-relativistic a Schro{\"o}dinger equation [REF]. In this way, the complexities of the full quantum field theory are reduced to a much more tractable problem.

This thesis presents a new prescription for parametrising the static heavy-quark potential in a background of hot and deconfined charge carriers, such as the QGP. By generalising the Gauss law of classical electromagnetism and combining this with a field-theoretic in-medium permittivity, the resulting in-medium complex potential admits an analytical solution. This can then be used to calculate spectral functions, and give realistic phenomenological predictions. The outline of this thesis is as follows: in Chapter 2, a short summary of some theoretical aspects of QCD is given, as well as how an introduction into quarkonium phenomenology both in vacuum and in the context of heavy ion collisions. Chapter 3 provides a detailed derivation of the in-medium potential and shows that this parametrisation is able to faithfully reproduce lattice data by utilising only one fitting parameter, the inverse screening length. Chapter 4 outlines the procedure with which phenomenologically relevant quantities such as the melting temperatures, decay widths, and double ratios can be calculated. The main results of this thesis are also given here, and a comparison is made with recent experimental results. A summary and outlook in given in Chapter 5. Appendix A contains a short introduction to thermal field theory and in particular the notion of a spectral function. Appendix B gives a more formal derivation of the Debye mass at one-loop order via Euclidean thermal field theory and finally, Appendix C shows how the structure of the in-medium permittivity arises from the Schwinger-Keldysh formalism. 

\chapter{Theory overview}
\onehalfspacing
\chapter{The in-medium potential}
\onehalfspacing
\chapter{Application to Heavy Ion Collisions}
\onehalfspacing
\chapter{Conclusion}
\onehalfspacing

\begin{appendices}
\chapter{Debye mass derivation}
\chapter{Schwinger-Keldysh formalism}
\end{appendices}

\bibliography{references}

\chapter*{Acknowledgements}


\cleardoublepage
\setlength{\parindent}{0em}

Erkl\"{a}rung:\par
\vspace{3\baselineskip}
Ich versichere, dass ich diese Arbeit selbstst\"{a}ndig verfasst habe und keine
anderen als die angegebenen Quellen und Hilfsmittel benutzt habe.\par
\vspace{5\baselineskip}
Heidelberg, den (Datum)\hspace{3cm}\dotfill


\end{document}
